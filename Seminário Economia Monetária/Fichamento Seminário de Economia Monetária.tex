% ----------------------------------------------------------------------------------------------------- %
% Manual da Classe UFTeX
% 
% Versão 2.1:   Março 2018
%
% Criado por:   Tiago da Silva Almeida
% Revisado por: Tiago da Silva Almeida
%               Rafael Lima de Carvalho
%               Ary Henrique Morais de Oliveira
%
% https://almeidatiago.github.io/uftex/
% ----------------------------------------------------------------------------------------------------- %

\documentclass[report]{uftex}
% ---- Esse comando cria o nome uftex estilizado
\newcommand\uftex{UF\TeX}

\usepackage{lipsum}
\usepackage{tikz}
\usepackage{enumitem}
\usepackage[siunitx]{circuitikz}
\usetikzlibrary{arrows}

\usepackage[alf,abnt-emphasize=bf]{abntex2cite}
\renewcommand{\backrefpagesname}{}
\renewcommand{\backref}{}
\renewcommand*{\backrefalt}[4]{}
% ----  Esse comandos são necessário no pré-ambulo para a impressão da lista de lista abreviatuas e de símbolos
\makelosymbols
\makeloabbreviations
% ---- Início do documento
\begin{document}
  % ---- Descrição do título do trabalho 
  \title{Fichamento: Uma breve
  	revisão do Novo Consenso Macroeconômico
  	a partir da crise de 2007/08 : questionamentos internos
  	acerca da condução da política econômica pós-crise}
  % ---- Nome do autor ou autores do trabalho
  \author{Lorenzo}{Costa Miranda}
  % ---- Nome do orientador do trabalho. O último campo representa o título do professor
  % ---- Departamento representa o curso ao qual o trabalho está sendo apresentado. Descrito por meio de duas iniciais do curso
  \department{CE}
  % ---- Data da apresentação do trabalho
  \date{08}{06}{2024}
  % ---- Palavras-chaves em português do trabalho
  \keyword{\LaTeX}
  \keyword{\uftex}
  \keyword{Trabalho de Conclusão de Curso}
  \keyword{Redação Científica}
  \keyword{Extensão Universitária}
  % ---- Palavras-chaves em inglês do trabalho
  \foreignkeyword{\LaTeX}
  \foreignkeyword{\uftex}
  \foreignkeyword{Bachelor Thesis}
  \foreignkeyword{Scientific Writing}
  \foreignkeyword{University Extension}
  % ---- Comando responsável por criar a capa do trabalho e/ou folha de resto conforme a configuração exigida
  \maketitle

  \frontmatter

  \printlosymbols  
  \printloabbreviations
  % ---- Cria a lista de figuras. OPCIONAL
  %\listoffigures
  % ---- Cria a lista de tabelas. OPCIONAL
  %\listoftables 
  % ---- Cria o sumário. OBRIGATÓRIO
 
% --- Marca o inicio dos elementos textuais. Capítulos.
\mainmatter
% ---- Defino o espaçamento de um e meio centímetros
\onehalfspacing
% ----------------------------------------------------------------------------------------------------- %
% Capítulos do trabalho
% ----------------------------------------------------------------------------------------------------- %
\chapter{Fichamento}

\section{Objetivo do Artigo}

Analisar os debates pós-crise do \textit{sub-prime}, e verificar as concordâncias e discordâncias quanto as propostas e revisões do Novo Consenso Macroeconômico no âmbito da política econômica. Evidenciar a controvérsia da crença de que o regime de política econômica adotado pelo Novo Consenso Macroeconômico garantiu a estabilidade entre os anos de 1990 e 2000 a partir da crise de 2008 e a prova de que essa longa recessão foi de responsabilidade das práticas impostas pelo regime.

\section{Introdução}

Os questionamentos da síntese neoclássica, fortificada por Friedman, e consolidadas pelos Novos Clássicos tomaram força nos anos 70, contaram com a eminência da crise do petróleo (1973 e 1979) e com o fim do Breton Woods (1973) no qual ocasionaram em impactos recessivos e inflacionários, além de elevadas taxas de juros nas décadas de 70 e 80 nos EUA e na Europa. Questionaram-se a curva de phillips e a eficácia da política discricionárias, ou seja, livre de restrições. A nova orientação vinda pelo \textit{mainstream} defendeu a volta do pensamento clássico para lastrear a trajetória da política econômica. 

Assim, o Novo Consenso Macroeconômico (NCM), em outras palavras Nova Síntese Neoclássica, estudou as práticas de diversos Bancos Centrais a partir do final de 80 em diversos países. Incorporaram-se a corrente Novo-Keynesiana, que ampliou o desenvolvimento teórico e metodológico desse novo consenso, contribuições dos neoclássicos, monetaristas, novos-clássicos e dos modelos de clico real dos negócios. Dessa forma, a nova síntese buscou justificar papel irrelevante da política fiscal em detrimento da política monetária, assim como retomar a ideia da neutralidade da moeda no longo prazo, abraçar o pressuposto das expectativas racionais e aceitação da curva de Lucas.

Mercados de concorrência imperfeita, fricções de preços e salários e a regra de Taylor, que substitui o mercado de LM, foram adicionados ao corpo teórico também.

A consequência desse conjunto de teses e teorias foi à implantação e consolidação do Regime de Metas e Inflação (RMI), que representou um conjunto de orientações teóricas de como a política monetária deveria ser conduzida por um Banco Central. Dessa, forma, autores passaram a apelidar o que estava se construindo como uma espécie de ``ciência monetária". Contudo, a crise financeira imobiliária que eclodiu nos EUA em 2007, e foi até 2008, demonstrou que em um período de estabilidade, existiu um processo que expôs a economia e o mercado financeiro ao risco.

De contra a uma estabilidade vivida pela economia norte americana, estava uma grande crise desenvolvida por um movimento especulativo do mercado financeiro, particularmente no mercado de hipoteca, cuja interligação com os diversos outros mercados generalizou os impactos nos setores do país. Como no caso da crise de 1029, políticas fiscais e principalmente monetárias em grande vulto, e até mesmo não convencionais, foram imprescindíveis para retardar os impactos da quebra financeira do país.

Dada essa resseção econômica, o NCM passou a sofrer críticas por causa das suas limitações quanto à condução da política econômica, e precisou ser renovada no \textit{pós-subprime}. Portanto, formularam-se duas principais hipóteses na reavaliação das teorias:

\begin{itemize}
\item[(i)] As mudanças na condução da política econômica são marginais, não afetando a essência do NCM;

\item[(ii)] As políticas monetárias não  convencionais são consideradas instrumentos para períodos de ruptura econômica, mas não para períodos nos quais a economia opera de forma eficiente. 	
\end{itemize}

\section{Método}

A partir da teoria e do conjunto de pressupostos e proposições iniciais do Novo Consenso Macroeconômico, busca-se mostrar as novas abordagens e as mudanças de paradigmas inicialmente postos. Procura-se também fazer uma análise comparativa entre os diversos autores sobre as novas proposições e críticas a partir da revisão do NCM. Porém, ainda, é preciso evidenciar todo esse arcabouço teórico-conceitual. 

O Novo Consenso Macroeconômico é um movimento que agrupa diferentes escolas econômicas e entre seus aspectos centrais está a hipótese de expectativas racionais, a relevância da demanda agregada na condução da atividade econômica no curto prazo, devido a rigidez de preços e salários micro fundamentadas, e a imperfeição de mercado em relação às taxas de crescimento no longo prazo. 

A partir dos pressupostos e da metodologia de equilíbrio geral estocástico, de acordo com a crítica de Lucas, o modelo adere a forma de uma curva de phillips e uma função IS. A instrumentalidade da política monetária é representada pela regra de Taylor, a inflação enquanto ancora nominal e a taxa de juros e a comunicação como as metas operacionais, que são passível de alteras as metas intermediárias (taxas de juros e longo prazo e expectativas de inflação) e deste modo à demanda agregada e consequentemente a inflação. 

De acordo com os novos clássicos e seus pressupostos como expectativas racionais, taxa natural de desemprego, inflação como fenômeno monetário, concorrência perfeita e flexibilidade dos preços levam a uma inconsistência temporal da política monetária discricionária, ou seja, livre de restrições, e resulta na ideia de um Banco Central independente, assim como a Curva de Lucas reforça esse ponto.

Segundo os Novos Keynesianos, os quais pressupõem expectativas racionais, taxa natural de desemprego, inflação como fenômeno monetário, concorrência imperfeita e rigidez de preços no curto prazo, leva a uma curva de phillips com rigidez nos preços, a uma IS ``foward looking", que ancoram o regime de metas de inflação (RMI)

Por último, o ciclo reais de negócios que descreve modelos de equilíbrio geral estocásticos levam à regra de Taylor e a uma ancora nominal e instrumento de política monetária (taxa de juros e comunicação)

A curva de phillips com rigidez de preços não reage a mudanças na taxa de crescimento de oferta de moeda, a não ser que este afete as expectativas de inflação e consequentemente da taxa real de juros; A curva de Taylor substitui a curva de Lm; O RMI deverá reagir a choques de demanda, permanecendo inalterado quando ocorre choques de oferta; a inflação se deve à indeterminação do produto potencial atrelado à taxa natural de desemprego, e da taxa de juros de equilíbrio

Os princípios da NCM: 1) a inflação é sempre e em toda parte um fenômeno monetário; 2) a estabilidade de preços traz benefícios importantes; 3) não existe compromisso de longo prazo entre desemprego e inflação; 4) as expectativas desempenham um papel crucial na determinação da inflação e na transmissão da política monetária à macroeconomia; 5) as taxas de juros reais precisam subir com o aumento da inflação, ou seja, o Princípio de Taylor; 6) a política monetária está sujeita ao problema da inconsistência temporal; 7) a independência do banco central ajuda a melhorar a eficiência da política monetária; 8) o compromisso com uma âncora nominal forte é fundamental para produzir bons resultados de política monetária (…). (MINSHKIN, 2011, p.)

Em momentos de recessão econômica, o instrumento taxa de juros perde sua viabilidade e sua eficácia quando se aproxima de zero, dada as expectativas, e as políticas não convencionais entram em ação para a retirada desse estado econômico. Consequentemente, atestou-se um forte aumento do endividamento público.

Entre diversos desacordos, o debate dos economistas mainstream se volta para uma posição intermediária como a principal referência para a condução da política econômica pós-crise. Muita é a crença de que o arcabouço teórico pré crise deveria permanecer, já que nenhum é invalidado pelos acontecimentos e a estratégia da política monetária continua igualmente válida. 

Ou seja, essas políticas monetárias não convencionais durante a crise não abalaram a estrutura teórica do modelo, mas para outros autores, ela apresentou vulnerabilidade e limitações na políticas. A discordância entre os autores repousa em relação às proposições aditivas à política monetária, tais como maior flexibilização do regime de metas de inflação, representadas pela necessidade de essas políticas interagirem com as macro prudenciais, controle de capitais e câmbio, assim como a redefinição  do papel da política fiscal.

\section{Contribuição do Artigo}

Em relação aos elementos teóricos, tem-se a concretização e reafirmação das hipóteses com a taxa natural de desemprego, expectativas racionais e inflação como um fenômeno monetário e fiscal. As alterações focam nos questionamentos ao âmbito operacional da política monetária, regulamentação financeira, política fiscal e política cambial. 

No geral, destaca-se a flexibilização na execução da política eocnômica - RMI mais flexível com novos objetivos, como a estabilidade financeira, utilização de PMNC e da política fiscal para tanto. Adoção de regulamentação macroprudencial e busca por maior espaço fiscal para as políticas anticíclicas nas recessões, uso de câmbio administrada e controladas para os países emergentes.

\subsection{Política Monetária}

A crise financeira demonstrou que uma meta de inflação baixa (por exemplo, de 2\% a.a) tem consequências danosas na efetividade da política monetária anticíclica. Por causa disso, alguns economistas \textit{mainstream} defendem que a meta da taxa de inflação poderia ser elevada (4\% a.a) em comparação ao período que aconteceu a recessão, o que permite uma ampliação da eficiência dessas políticas ao operar a taxas de juros de curto prazo, mas teria que avaliar os custos das distorções geradas em prol da maior efetividade da política monetária convencional. São elas: 

\begin{itemize}
	\item[(i)] Ancorar expectativas em 4\% a.a é mais difícil;
	\item[(ii)] Maior volatilidade da inflação, decorrente do sistema fiscal que não é neutro a inflação;
	\item[(iii)] Impacto sobre o saldo real de moeda;
	\item[(iv)] Mudanças estruturais na economia.
\end{itemize}

Uma meta de inflação acima de 3\% promoverá uma tendência de elevação da inflação para patamares mais altos e a escolha de um único índice para descrever a inflação é um problema quando o índice de inflação é comparado ao produto ou aos preços dos ativos e agregados monetários, pois pode haver a possibilidade de a inflação estar estável, porém o hiato do produto não, bem como a inflação e o produto estarem estáveis, mas o comportamento de preços de ativos e agregados monetários não. 

Taylor questiona que tais medidas em um contexto de aumento da dívida pública e expansão do balancete do BC poderiam diminuir consideravelmente a confiança da meta de inflação, e consequentemente, a independência do BC, e tais medidas seriam ainda mais inadequadas se adotadas pelos países emergentes, que apresentam taxas de inflação normalmente mais elevadas. 

O terceiro ponto da revisão é sobre a conduta da política monetária para controlar o preço dos ativos financeiros. Alguns consideram a taxa de juros pouco sensível às variações de preços dos ativos, de modo que teria pouca efetividade para controlar o excesso de alavancagem das instituições financeiras, elevações de riscos de mercado e valorização dos preços dos ativos. Ademais, aceitava-se a ``doutrina greenspan”, que afirma que o BC deveria atuar somente após a ruptura da bolha, porém, outros autores afirmavam que o melhor seria ``inclinar-se contra o vento”, ou seja, uma política de prevenção. 

A crise volta a discussão sobre a política monetária ser um instrumento para mitigar as bolhas, com a justificativa de que é extremamente complicado diferenciar aumento de preços de um mercado específico como mudanças nos fundamentos básicos de oferta e demanda, ou surgimento de bolhas especulativas. Mesmo que as taxas de juros fossem pouco sensíveis às variações de preços dos ativos, e a utilização de taxa de juros mais baixas implicaria em riscos mais elevados, e um possível aumento dessas taxas, com o fito de reduzir os riscos, levaria a um aumento do hiato do produto prejudicando o produto e o emprego. E por isso, a melhor resposta seria a utilização de política macroprudencial (visa mitigar os riscos financeiros). 

(Mishkin) afirma que a política monetária crível suscita a redução de incerteza, ou seja, para mitigar o risco macroeconômico, no qual o BC deverá atuar de forma preventiva à ocorrência de rupturas financeiras com mudanças na taxa de juros de curto prazo. Essa proposta é reafirmada por (Woodford), que propõe uma nova meta a RMI com a finalidade da estabilidade financeira. Dessa forma, o BC poderá manter o compromisso com a meta de inflação de longo prazo, porém com a possibilidade de alterar a taxa de juros de curto e médio prazo para prevenir as crises.

Porém, Taylor discorda desses autores em relação tanto à regulamentação macroprudencial, quando do gerenciamento do risco. Ele defende que a política monetária deveria continuar usando a regra de condução de Taylor, como foi durante a década de 1990 o que garantiu a estabilidade do nível de preços, do produto e das condições financeiras por todo esse tempo. Porem, no precedente da crise de 2008, a taxa de juros de curto prazo estava baixíssima em relação à recomendação da regra de Taylor, o que ampliou o crédito ao setor imobiliário aumentando o risco. Segundo o autor, o risco decorre dos impactos indesejados de taxas de juros muito baixas, no qual nesse caso, a autoridade monetária deveria seguir com a regra a mitigar riscos. 

Outra questão importante é a política de sinalização, ou gestão de expectativas, propostas por Eggertsson e Woodford, e questionada por Willians, no qual afirma que a "orientação para frente” é inconsistente temporalmente. Isso porque o BC tem compromisso de manter a taxa de juros de curto prazo inferior ao que faria no futuro, permitindo uma inflação mais elevada a frente. Mas o BC poderia não cumprir esse compromisso e elevar as taxas de juros para conter a elevação da taxa de inflação, e essa inconsistência poderia ter um impacto negativo quanto às expectativas sobre a credibilidade.

Outros autores concordam com esse argumento, mas que para evitar o problema temporal, a comunicação do BC com o público deverá explicitar a condução da política monetária, essa com discricionariedade restrita, em relação às taxas de juros e que dependem das condições econômicas e poderiam ser alteradas de acordo com as necessidades. Porém, Clarida não aceita essa posição alegando isso pode ser percebido pelo público, comprometendo a ancoragem da inflação no futuro. Como alternativa, Woodford apresentou uma proposta de meta de PIB nominal calculado pela multiplicação do hiato do produto e de uma meta de inflação.

\subsection{Política Financeira}

No que tange à política financeira, que vem passando por desregulamentos nos aspectos relacionados à movimentações financeiras desde 1970, sofreu um certo abandono da regulamentação e supervisão como uma ferramenta macroeconômica, sobre argumento da irrelevância teórica da intermediação financeira. 

Para reforçar a separação entre a política monetária e a política financeira no regime de metas inflacionárias, foi atribuída o "princípio de Tinbergen", no qual afirma que para o atingimento de cada objetivo de política econômica deve ser utilizado um único instrumento. Dessa forma, a política monetária ficou responsável pela estabilidade do nível de preços de bens e serviços de modo que a política financeira foi substituída pela regulamentação microprudencial das instituições depositárias, que poderiam afetar a atividade econômica por meio do canal de crédito (minimização dos riscos através dos depósitos compulsórios e empréstimos de liquidez). 

Entretanto, dada essa desregulamentação, sustentada pela hipóteses de mercados eficientes, permitiu que as instituições não depositárias estivessem de fora de qualquer forma de regulamentação e supervisão microprudencial. Ou seja, essa nova estrutura regulatória, cujo foco era microprudencial e dirigida ao mercado bancário, mostrou a vulnerabilidade da política monetária no momento de pré-crise, isto é, nenhuma instituição tinha poder de adotar medidas macroprudenciais de redução de risco. Segundo Blanchard - a regulamentação microprudencial não leva em conta as interações entre as instituições financeiras, o setor financeiro e a economia real -.

Dessa forma, Mishkin constrapõe a ``doutrinha Greenspan", afirmando que a crise demonstrou que os custos de atuação do BC após o estouro da bolha de crédito são muito mais expressivos que o custo da ação preventiva. Mas a discussão centra em qual instrumento é de fato o mais adequado para a intervenção. 

Mikshin ainda afirma que a regulamentação microprudencial é ainda um conjunto de regras que deve permanecer como um dos arcabouços regulatórios, mesmo que não seja o ideal para as situações de falha de mercado. Por outro lado, as políticas macroprudenciais são relevantes para regular as interações entre as instituições e o mercado financeiro. Reforçando, para situações de falha de mercado, o mais apropriado é a revisão macroprudencial. Dessa forma, aponta-se para a quebra da dicotomia entre a política monetária e a política financeira, provando que as duas são inseparáveis para a estabilização inflacionária, do produto e financeira. 

 
O RMI não precisa ser reprovado como estrutura de política econômica, mas deve ser complementado por outros instrumentos, rejeitando assim o princípio de Tinbergen. O ideal é coordenar a política monetária e a regulamentação macroeconômica prudencial com a política financeira, em que a comunicação com a sociedade influencie na sua expectativa na direção pretendida do governo. Essa responsabilidade deve ser auferida totalmente ao BC. 

Porém, essa posição não é aceita por Taylor, que critica a utilização de políticas regulatórias para lidar com booms de crédito ou preços dos ativos e defende a regulamentação convencional. Para ele, a taxa de juros adequada é o principal instrumento para impedir as situações de crise, e a utilização de políticas financeiras promove a perda de independência do BC.

\subsection{Política Fiscal}

Inicialmente, a política fiscal foi subestimada, justamente pelas críticas dos monetaristas e novo-clássico por causa dos efeitos "crowding out" e pela equivalência ricardiana, que reduziu a prioridade da política fiscal em detrimento de uma regra de controle orçamentário com o fito de dar suporte e estabilidade de preços. Porém, verificou-se que na crise, essas políticas foram um importante instrumento anticíclico em momentos de taxas de juros muito baixas. Primeiro, Foi a única alternativa ainda disponível para estimular a economia, dado que as políticas monetárias convencionais e não convencionais estavam em seu limite.  

Multiplicadores fiscais se sobressaem quando a política monetária alcança níveis próximos ou abaixo de zero, no geral quando a economia se encontra em recessão. Segundo, dadas as expectativas baixas em relação a crise, pouco havia de forças críticas a essas políticas.

Entretanto, muitas nações, com alta restrição orçamentária intertemporal, não optaram por liquidar os gastos governamentais com cobrança de impostos futuros, mas sim com a emissão de passivos monetário pelo BC, elevando a inflação de qualquer forma. A opção pela monetização da dívida na compra de títulos do tesouro por meio do BC, resulta na sua perda do controle de preços, no qual também gera inflação. Em todos os casos que envolvem default da dívida por parte do governo, ou seja, calote ou moratória, a autoridade monetária não consegue a estabilidade de preços. Mesmo se não houvesse esse fenômeno, é questão de tempo para ocorrer a monetização da dívida decorrente das dívidas.

Ou, em contrapartida, caso não houver a monetização de nenhuma forma, ocorrerá uma elevação nas taxas de juros e contração da atividade econômica, no qual aumentaria os riscos de default dos títulos e promoveria uma ruptura financeira. %Princípios macroeconômicos conhecidos.

Segundo Mishkin, a crise gerou um aumento do déficit público, e em algum momento a política monetária será convocada para monetizar a dívida pública causando inflação, e se por acaso não seja, também levará à inflação. Ele também argumenta que no pós-crise, a política fiscal deverá promover uma redução do nível de endividamento e manter-se no orçamento. 

Blanchard defende a revisão do papel da política fiscal para instrumentos anticíclico, e nos momentos de prosperidade e crescimento econômico deverá prezar pela redução da dívida/PIB, preservando espaço para os gastos e incentivos nos momentos de recessão econômica. Mas ainda reforça o limite da política e o compromisso com o orçamento no médio e longo prazo, e apenas flexibilidade em períodos de crise.

\subsection{Política cambial}

A crise também resultou em grande volatilidade no fluxo de capitais nas economias, principalmente naquelas em que havia abertura da conta de capitais, atividade econômica menos diversificada e mercados financeiros mais densos, economias em desenvolvimento no geral. Com a derrocada do fluxo de capitais nas economias abertas, fortes pressões na taxa de câmbio surgiram.

Até a crise, os países desenvolvidos que utilizavam o RMI adotavam o câmbio flexível e a política cambial só de fato importava na medida em que afetava a meta de inflação. Assim, as variações na taxa de câmbio nominal, que promovesse desvalorização da moeda nacional, eram preocupações para o BC. Porém, nos países onde os mercados financeiros são integrados, os instrumentos cambiais acabam por serem supérfluos, pois os fluxos de capitais tendem a reagir aos diferenciais de taxas de juros. 

Entretanto, nos países emergentes, onde as oscilações cambiais, decorrente da variação do fluxo de capitais, promoveram impactos na estabilidade financeira e macroeconômica em que a taxa de câmbio teve que ser gerenciada pelo BC a partir de intervenção no mercado cambial. Porém Ostry, diz que essas políticas descredibilizarão a política monetária, a não ser que os BCs reconheçam publicamente que a estabilidade do câmbio é em prol de maximizar o bem-estar. Assim, autores defendem que a meta de taxa de câmbio real e a intervenção estatal são um segundo componente do RMI para lidar com as consequência da alteração dos fluxos de capitais. 

Variações significativas em relação ao fluxo de capitais poderiam ser compensadas por alterações nas taxas de juros de curto prazo, além da intervenção do BC no mercado de câmbio doméstico. Após a crise de 2008, o FMI reviu sua posição quanto ao controle de capital e propôs uma nova visão institucional no qual aceitou controles de capitais em circunstâncias específicas dando mais espaço às economias emergentes

\subsection{Revisão dos Modelos DSGE - Modelo Dinâmico Estocástico de Equilíbrio Geral}

Os novos aspectos metodológicos e princípios da nova síntese neoclássica. Mishkin questiona a validade de duas hipóteses normalmente adotadas.

A primeira vai de contra a teoria da política monetária ótima, a partir da maximização de um função linear quadrática dos desvios de inflação e do produto, por causa de os eventos da crise terem demonstrado que a economia não tem comportamento linear da distribuição de Gauss. A segunda questão é sobre a necessidade dos modelos DSGE incorporarem possibilidade de fricções financeiras e abandonarem a estrutura de agente representativo.

Portanto, no pós-crise, essa estrutura baseada em um agente padronizado tem sido deixada de lado, concebendo o fato de que esses captem ocorrências de fricções financeiras na crise é uma modelagem inapropriada acerca dos agentes econômicos. Woodford contribui com a ideia de tratar os eventos da crise como elemento endógeno que impacta a instabilidade financeira e a função política do BC, isto é, um fator a planejado.

\section{Conclusão}

De modo geral, essas novas proposições como um todo representam o relaxamento do NCM, ainda que não questionando os principais elementos teóricos. Porém, a visão que ainda prevalece no \textit{mainstream} é a de manutenção dos elementos do NCM como a estabilidade de preços como objetivo primário de longo prazo e a garantia de um BC independente para alcançar esse propósito, e reforça a tecla da importância da política monetária para os períodos de recessão financeira e econômica.










\end{document}
